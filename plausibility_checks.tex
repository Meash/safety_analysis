\documentclass[12pt,a4paper,draft]{article}
\usepackage[utf8]{inputenc}
\usepackage[german]{babel}
\usepackage{amsmath}
\usepackage{amsfonts}
\usepackage{amssymb}
\usepackage[LGRgreek]{mathastext}


\title{Adäquatheit der Modellierung des Sollverhaltens des FFBs ohne Störungen}
\date{\vspace{-5ex}} % blank date without spacing


\newcommand{\AF}{\ \textbf{AF}\ }
\newcommand{\AG}{\ \textbf{AG}\ }
\newcommand{\AU}{\ \textbf{AU}\ }
\newcommand{\EF}{\ \textbf{EF}\ }
\newcommand{\EU}{\ \textbf{EU}\ }
\newcommand{\myeq}[1]{%
% 1: content
\begin{equation}
\begin{split}
#1
\end{split}
\end{equation}
}


\begin{document}

\maketitle


\section{Safety} 
\myeq{
&\neg \EF (Zugposition.PositionReal = 0 \land \neg (Bahnuebergang = Gesichert)) \\
&\AG (Zugposition.PositionReal = 0 \Rightarrow Bahnuebergang = Gesichert)
}
\myeq{
\AG Zugposition.&PositionReal = 0 \\
	& \Rightarrow Bahnuebergang = Gesichert \\
	& \AU Zugposition.PositionReal = Sensorpunkt
}

\section{Bahnuebergang}
\myeq{
\AG (&Bahnuebergang = Gesichert \\
	& \Rightarrow \AF Bahnuebergang = Ungesichert)
}

\subsection{Schranke}
\begin{equation}
\AG (Schranke.Winkel = 0 \Rightarrow Schrankensensor = Geschlossen)
\end{equation}
\begin{equation}
\AG (Schranke.Winkel = 2 \Rightarrow Schrankensensor = Offen)
\end{equation}

\myeq{
\AG (&Schranke.Winkel = 2 \Rightarrow \\
	& \AF Schrankenmotor = DrehrichtungSchliessend \\
	& \AU Schranke.Winkel = 0)
}
\myeq{
\AG (&Schranke.Winkel = 0 \Rightarrow \\
	& \AF Schrankenmotor = DrehrichtungOeffnend \\
	& \AU Schranke.Winkel = 2)
}

\subsection{Sensor}
\myeq{
\EF (&Zugposition.PositionReal >= SP \\
	& \land Zugposition.PositionReal <= SP + Messweite) \\
	& \Rightarrow Sensor = ZugFaehrtDurch \AU Sensor = KeinZug
}


\section{Funk}
\myeq{
\EF &ZugFunksender = SicherungsanweisungGesendet \\
	&\Rightarrow \AF BahnuebergangFunkempfaenger \\
    &= SicherungsAnweisungAngekommen
}
\myeq{
\EF &ZugFunksender = AnfrageGesendet \\
	&\Rightarrow \AF BahnuebergangFunkempfaenger \\
    &= ZustandsAnfrageAngekommen
}
\myeq{
\EF &BahnuebergangFunksender = BestaetigungsNachrichtGesendet \\
	&\Rightarrow \AF ZugFunkempfaenger = SicherungsBestaetigungErhalten 
}


\section{Zug}
\subsection{Bremse}
\begin{equation}
\AG (Zugbremse = Bremsend \Rightarrow \AF Zuggeschwindigkeit.vReal = 0)
\end{equation}

\subsection{Zugposition}
\myeq{
&Zugpositionsbestimmung = AufFreierStrecke \\
&\AU Zugpositionsbestimmung = EinschaltpunktErreicht \\
&\AU Zugpositionsbestimmung = AnfragepunktErreicht \\
&\AU Zugpositionsbestimmung = BremseinsatzpunktErreicht \\
&\EU Zugpositionsbestimmung = GefahrenpunktErreicht \\ % Zwangsbremsung
&\AU Zugpositionsbestimmung = Ende
}

\subsection{Zugsteuerung}
\myeq{
\AG (&Zugpositionsbestimmung = BremseinsatzpunktErreicht \\
	& \land \neg Bahnuebergang = Gesichert \\
    & \Rightarrow Zugsteuerung = Zwangsbremsung)
}

\end{document}
